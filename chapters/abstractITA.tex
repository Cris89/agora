\chapter*{Sommario}

\lettrine{A}{pplicazioni} che si interessano di campi di ricerca complessi come, ad esempio, gli studi sull'universo o sulla microbiologia, stanno proiettando le architetture ad elevate prestazioni di calcolo (\textit{High Performance Computing}) verso il traguardo di un miliardo di miliardi di operazioni effettuate al secondo.

Queste intricate ricerche sono caratterizzate da una moltitudine di dati in ingresso e dalla presenza di parametri che ne influenzano l'esecuzione; poichè problematiche inerenti il consumo di potenza e l'efficienza energetica hanno assunto un'importanza rilevante, esistono varie tecniche che tentano di migliorare questi aspetti, mantenendo comunque accettabile il valore dei risultati ottenuti: ad esempio, strategie di approssimazione (\textit{Approximate Computing}), applicabili sia a livello hardware sia a livello software, ricercano un equilibrio tra la qualità della computazione e lo sforzo richiesto, al fine di adempiere ai vincoli e agli obiettivi dell'applicazione e di massimizzare, al contempo, l'efficienza generale.

Per questo tipo di applicazioni, lo spazio delle possibili configurazioni è molto ampio e, pertanto, non è possibile esplorarlo esaustivamente: avere completa conoscenza riguardo le combinazioni dei parametri e i corrispondenti risultati delle metriche prese in esame (come, ad esempio, il numero di operazioni completate al secondo, il consumo di potenza oppure la precisione dei dati in uscita) è irrealizzabile.

Questa tesi ha sviluppato un sistema, Agorà, in grado di gestire l'esplorazione dello spazio delle configurazioni attraverso un sottoinsieme di esso, al fine di predire, attraverso tecniche di apprendimento automatico (\textit{Machine Learning}), il modello completo dei programmi; questo risultato è usato da un cosiddetto \textit{autotuner} per scegliere dinamicamente la migliore combinazione di parametri che soddisfa vincoli e obiettivi correnti dell'applicazione in gestione. I principali benefici apportati da Agorà sono l'eliminazione di ogni fase antecedente l'esecuzione dei programmi e la capacità di gestire molteplici applicazioni contemporaneamente.

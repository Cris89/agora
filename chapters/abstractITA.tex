\chapter*{Estratto}

\begin{otherlanguage}{italian}

\lettrine{A}{}\textit{pplicazioni} di ricerca avanzata, come gli studi sull'universo o sulla microbiologia, stanno proiettando le architetture ad elevate prestazioni di calcolo (\textit{High Performance Computing}) verso il traguardo di un miliardo di miliardi di operazioni effettuate al secondo, atteso nel 2022.

Queste ricerche sono caratterizzate da una enorme quantità di dati in ingresso e dalla presenza di parametri che ne influenzano l'esecuzione. Poichè problematiche inerenti il consumo di potenza e l'efficienza energetica hanno assunto un'importanza rilevante, esistono varie tecniche che tentano di migliorare questi aspetti, mantenendo comunque accettabile il valore dei risultati ottenuti. Strategie di approssimazione (\textit{Ap\-prox\-i\-mate Com\-put\-ing}), applicabili sia a livello hardware sia a livello software, cercano di raggiungere un equilibrio tra la qualità della computazione e lo sforzo richiesto, al fine di adempiere ai vincoli e agli obiettivi dell'applicazione e di massimizzare, al contempo, l'efficienza di calcolo.

Per questo tipo di applicazioni, lo spazio delle possibili configurazioni è molto ampio e, pertanto, non è possibile esplorarlo esaustivamente. Possedere una completa conoscenza riguardo le combinazioni dei parametri e i corrispondenti risultati delle metriche prese in esame (come, ad esempio, il numero di operazioni completate al secondo, il consumo di potenza oppure la precisione dei dati in uscita) è praticamente irrealizzabile e, pertanto, si ricorre a soluzioni approssimate.

Questa tesi ha sviluppato un sistema, \textit{Agora}, in grado di gestire l'esplorazione di un sottospazio delle possibili configurazioni, con lo scopo di predire, usando tecniche di apprendimento automatico (\textit{Machine Learning}), il modello completo dei programmi in esame; questo risultato è utilizzato da un \textit{autotuner} per sce\-glie\-re dinamicamente la migliore combinazione di parametri che soddisfi vincoli e obiettivi dell'applicazione. I principali benefici apportati da Agora sono l'eliminazione di ogni fase an\-te\-ce\-den\-te l'esecuzione dei programmi e la capacità di gestire molteplici applicazioni contemporaneamente.

\end{otherlanguage}
